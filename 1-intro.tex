%!TEX root = doc.tex
\section{Introduction} % (fold)
\label{sec:introduction}

% 1 - Intro to MAS
A \gls{MAS} is composed of autonomous intelligent agents, capable of interacting with each other \cite{wooldridge2008introduction}.
Agent-based applications are in widespread use in multiple fields of research and industry and can be heterogeneous, often requiring interoperation between agents from different systems.
In order to make this possible, agent technologies have matured and standards have emerged to support the interaction between agents.

% 2 - FIPA and ACL
The specifications of \gls{FIPA}\footnote{http://fipa.org/} promote interoperability in heterogeneous agent systems.
These standards define not only a common \gls{ACL}, but also a group of interaction protocols, recommended facilities for agent management and directory services \cite{o1998fipa}.

% 3 - Framework problems (not supporting FIPA)
Several frameworks exist \cite{survey,survey2} that offer some level of abstraction from low level development, allowing programmers to focus on a more conceptual approach in \gls{MAS} design.
% They are usually focused in one or more domains.
It turns out, though, that \gls{FIPA} standards (or any standards) are not supported by all of them.

% 4 - This paper focus JADE and Repast
In this paper we focus on two frameworks, the \gls{JADE} and the \gls{Repast}. For this paper, version 4.3.2 of JADE and version 2.1 of Repast Simphony (henceforth designated simply as ``Repast'') were used.

% 4.1 - \gls{JADE}
\gls{JADE} is a \gls{FIPA}-compliant, general-purpose (i.e. not focused on a single domain) framework used in the development of distributed agent applications.
It seamlessly hides all complexity concerning its distributed architecture \cite{bellifemine2003JADE}.
However, experiments with \gls{JADE} show that the platform's scalability is limited \cite{mengistu2008scalability}, meaning that \gls{JADE} is not an appropriate tool to create MABS which usually deal with a large number of agents.

% 4.2 - Repast
Repast, on the other hand, is a very rich simulation tool \cite{repSimph}, but does not support any open standards. Table \ref{tab:jadevsrep} summarizes the main differences between the two frameworks. One of the main motivations of the solution we propose is to enable Repast-based \gls{FIPA}-compliant simulations, thus making them more similar to distributed MAS implementations. This solution can be useful in scenarios where proficient JADE developers intend to create Repast simulations, allowing them to use familiar JADE-like tools. 
With this goal in mind, we wish to enable the conversion of Repast models into full featured \gls{JADE} MAS, as well as the conversion of \gls{JADE} MAS into Repast simulations, thus further closing the gap between the two frameworks.

\begin{table}[h]
	\normalsize
	\caption{Comparison of JADE and Repast features.}
	\label{tab:jadevsrep}
	\begin{center}
		\begin{tabular}{l|cc}
		\hline

		\hline
		\textbf{} & \textbf{JADE} & \textbf{Repast} \\ %& \textbf{Cougaar} \\
		\hline
			Agent 		& ACL  	&  Method calls  \\ %& Serialized Object \\
			Interaction	& Messages	&  Shared resources \\
		\hline
			Distribution & Yes & No \\ %& Yes \\
		\hline
			Simulation Tools & No & Yes \\ %& Yes \\
		\hline
			Scalability & Limited & High \\ %& High \\
		\hline
			Ontologies & Yes & No \\ %& Yes\\
		\hline
			Agent  		& Behavior-based & Schedule-based  \\ %&  \\
			Execution	& Multi-threaded & Single-threaded \\ %&  \\
						& Event-driven   & Tick-driven 	   \\ %&  \\
						& Assync		 & Sync 		   \\ %&  \\
		\hline
		\end{tabular}
	\end{center}
\end{table} 

% 5 - Proposal 
With these goals in mind, we propose the use of an API when developing Repast-based simulations. \apiname{} (\apilongname{}) addresses not only agent programming, but also their interaction and related infrastructural components.
We are aware that this kind of facilities is only valuable for true multi-agent based simulation, and not in general for any agent-based modelling and simulation approach (for which Repast is also suited).

% 6 - Challenges
Developing \apiname{} presented some challenges.
%Some of the challenges of developing this software lie in balancing its complexity.
MABS's execution is usually very fast paced; maintaining a simple architecture with a low impact on performance is therefore important.
The architecture we propose, however, is based on \gls{JADE}.
A compromise has been met by porting the features from \gls{JADE} that we find most valuable in a \gls{MAS} and that are needed for a meaningful exploitation of both frameworks.

% 7 - Objectives
One of the guidelines while developing this API was to make the software considerably generic, in the sense that it is not heavily dependent on a framework.
We accomplished this by using a pure Java implementation that does not rely on constructs provided by Repast or \gls{JADE}, as opposed to some of the similar works we discuss in Section \ref{sec:related_work}.
%% Related work
Those works try to integrate both frameworks in a single environment, seeking to use the best of \gls{JADE} and Repast. They are still relevant to our proposal since they give important insight on how to implement interaction protocols available in \gls{JADE} (inherently event-driven) in Repast's tick-based scheduler.

Section \ref{sec:fipa} contains a brief discussion about FIPA specifications, especially focused on the concepts present in JADE that are relevant to this project.
%% Architecture
The details of the API are defined in Section \ref{sec:proposal}. We provide a clear overview of its architecture and justify our design decisions and how it compares to \gls{JADE}'s own implementation.
%% Verification
For exemplification and comparison purposes, in Section \ref{sec:verification} we present an experiment performed in both frameworks and using \apiname{}.
%% Prototype
In Section \ref{sec:prototype} we present our proposal for the code conversion tool that will allow programmers to convert between Repast simulations created with \apiname{} and \gls{JADE} applications. While still under development, we try to give an overview of its main features.
% Conclusions
Finally, Section 7 concludes the paper and points future work.







