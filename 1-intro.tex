%!TEX root = doc.tex
\section{Introduction} % (fold)
\label{sec:introduction}

Multi-agent systems (MAS) are composed of autonomous computational elements, called agents, capable of interacting with each other \cite{wooldridge2008introduction}. Agent-based applications and simulations are in widespread use in multiple fields of research and industry.  and can be very heterogeneous, often requiring interoperation between agents from different systems.  Therefore, for agent-based technologies to mature, standards needed to emerge to support the interaction between agents.

The specifications of the Foundation for Intelligent Physical Agents (FIPA\footnote{The Foundation for Intelligent Physical Agentsh - http://fipa.org/}) include such standards and try to promote interoperability in heterogenous agent systems. These standards define not only a common agent communication language (ACL), but also a group of interaction protocols, recommended facilities for agent management and directory services \cite{o1998fipa}.

Several frameworks exist that offer some level of abstraction from low level development, allowing programmers to focus on a more conceptual approach in MAS design. They are usually focused in one or more domains. As it turns out, though, FIPA standards (or any standards) are not supported by all of them \cite{survey}.

In this paper we focus on two frameworks, JADE and Repast Simphony. JADE is a FIPA-compliant, general-purpose (i.e. not focused on a single domain) framework used in the development of distributed agent applications. JADE seamlessly hides all complexity concerning its distributed architecture. \cite{bellifemine2003jade} However, experiments with JADE show that the platform's scalability is limited, meaning that JADE is not the best tool to create MABS which usually deal with a large number of agents. \cite{mengistu2008scalability} \cite{garcia2011misia}

Repast, on the other hand, is a toolkit focused in the creation and execution of Multi-Agent-Based Simulations (MABS). Repast S is a very rich simulation tool, but doesn't support any open standards. \cite{collier2003repast} As such, one of the main motivations for the creation of the software described in this paper is to enable Repast-based FIPA-compliant simulations to be created, thus improving its interoperability with JADE. As an ultimate goal, we wish to enable the conversion of Repast models into full featured JADE MAS, as well as the conversion of JADE MAS into Repast simulations.

The first step towards our goal is to implement FIPA interaction protocols in Repast. We this goal in mind, we propose the use of an API when developing Repast-based simulations. The software, entitled \apiname{} defines not only the agents, but also their interaction behaviors and agent management facilities.

Some of the challenges of developing this software lies in balancing its complexity. MABS's execution is usually very fast paced; maintaining a simple architecture that reduces its impact on performance is therefore important. However, in order to enable interoperability and the possibility of conversion, the architecture we propose was based in JADE. A compromise was met by porting the features from JADE that were essential to performing agent communication.

One of the achievements while developing this API was to make the software somewhat generic, in the sense that it was not heavily depended on a framework. We accomplished this by using pure a Pure Java implementation that doesn't rely on constructs provided by Repast or JADE, as opposed to some of the similar works we discuss later in this paper.

In section \ref{sec:related_work}, we analyze some related works where the authors tried to integrate both frameworks in a single environment, enabling them to use the best of JADE and Repast. These works are still relevant to our proposal since they give important insight on how to implement the interaction protocols from JADE (inherently event-driven) in Repast's tick-based environment.

The details of the architecture, protocols and usage guidelines are defined in section \ref{sec:proposal}. We give a clear overview of the details of the current implementation, while justifying the design decisions that were made and showing how it compares to JADE's own implementation. As of the writing of this paper, \apiname{} is capable of being used in the creation of MABS using three kinds of FIPA interaction protocols: `Request-like', Contract Net and Propose.

In section \ref{sec:prototype} we present our proposal for the code conversion tool that will allow us to convert between Repast simulations created with \apiname{} and JADE applications. It's still an early prototype, but in section \ref{sec:verification} we present some usage examples and results of our implementation.